%
% To start the document, use
%  \levela{...}
% For lover level, sections use
%  \levelb{...}
%  \levelc{...}
%
\levela{Basic radiative transfer}
 \label{sec:rte}


%
% Document history, format:
%  \starthistory
%    date1 & text .... \\
%    date2 & text .... \\
%    ....
%  \stophistory
%
\starthistory
  000307 & Started by Patrick Eriksson. \\
  000908 & First version finished by Patrick Eriksson. \\
  031205 & Cooling rates added by Patrick Eriksson. \\
\stophistory



%
% Symbol table, format:
%  \startsymbols
%    ... & \verb|...| & text ... \\
%    ... & \verb|...| & text ... \\
%    ....
%  \stopsymbols
%
%
%\startsymbols
%  \mpbi     & \verb|y|       & monochromatic pencil beam intensity      \\
%  $\mpbi_1$ & \verb|y_cbgr|          & cosmic background radiation      \\
%  $\mpbi_1$ & \verb|y_ground|        & ground blackbody radiation       \\
%  $S$       & \verb|S|               & source function                  \\
%  $k$       & \verb|Abs|             & total absorption                 \\
%  $T$       & \verb|t|               & temperature                      \\
%  \f        & \verb|f|               & frequency                        \\
%  $h$       & \verb|PLANCK_CONST|    & the Planck constant              \\ 
%  $c$       & \verb|SPEED_OF_LIGHT|  & speed of light                   \\
%  $k_B$     & \verb|BOLTZMAN_CONST|  & the Boltzmann constant           \\
%  $\tau$    & -                      & optical thickness                \\
%  $\zeta$   & \verb|Tr|              & transmission between LOS points  \\
%  $\zeta^{atm} $& \verb|y|           & total atmospheric transmission   \\
%  $l$       & \verb|l|               & distance along LOS               \\
%  $\Delta l$ &\verb|lstep|           & step length along LOS            \\
%  $e$       & \verb|e_ground|        & ground emissivity                \\
%  $T_{ground} $&\verb|t_ground|      & physical temperature of the ground\\
%  $T_{cbgr}$  &\verb|COSMIC_BG_TEMP| & temperature representing cosmic  \\ 
%         &                &        background radiation                 \\
% \label{symtable:rte}     
%\stopsymbols



%
% Introduction
%
This section discusses the solution of the atmospheric radiative
transfer equation (RTE). A non-scattering atmosphere in local
thermodynamic equilibrium is assumed. 
The radiative transfer equation gives the
monochromatic (infinite frequency resolution) pencil beam (infinite
spatial resolution) spectrum. The main problem here is
how to practically and accurately estimate the (continuous) integral
in the discrete forward model.
 
The discussion treats mainly measurements of atmospheric emission. The
forward model can also handle pure absorption measurements (that is,
emission is neglected) and such observations are discussed last in the
section.

The equations of this section are valid for monochromatic pencil beam
spectra, no effects of the sensor are considered. How to
incorporate sensor effects in the spectra is discussed separately
(Sec. \ref{sec:sensor}).



\levelb{Introduction}
 \label{sec:rte:intro}
 
 Atmospheric radiative transfer can be expressed generally as
 \begin{equation}
   I = I_1e^{-\int_{l_1}^{l_2}{\kappa(l)dl}} +
        \int_{l_1}^{l_2}{\kappa(l)\sigma(l)e^{-\int_{l}^{l_2}{\kappa(l')dl'}}dl}
    \label{eq:rte:rte}
 \end{equation}
 where $I$ is the monochromatic pencil beam intensity, $l$ distance
 along the line of sight (LOS), $l_1$ the point of the considered part
 of the LOS furthest away from the sensor, $l_2$ the closest point of
 the LOS, $I_1$ the intensity at $l_1$, $\kappa$ the total absorption
 along the LOS and $\sigma$ the source function.\footnote{The symbols
   $\kappa$ and $\sigma$ are used here for the absorption and the source 
   function \emph{along} the LOS. The more commonly used symbols, $k$ and
   $S$, respectively, are used below to express the variables as
   functions of altitude.}
  
 Equation \ref{eq:rte:rte} is of general validity if $\sigma$ and $\kappa$
 consider the relevant effects, for example, scattering. However, below in
 this section it is assumed that there is no scattering and the
 atmosphere is in local thermodynamic equilibrium.
  
 Note that Eq. \ref{eq:rte:rte} is valid both for the case when the LOS is
 determined by geometrical calculations and when refraction is
 considered (the refraction changes however the LOS).
  
 With the assumptions of no scattering and local thermodynamic
 equilibrium, $\kappa$ is the summed gaseous absorption, and the source
 function equals the Planck function, $B$:
 \begin{equation}
    \sigma = B(\f,T) = \frac{2h\f^3}{c^2} \frac{1}{e^{h\f/k_B T}-1}
    \label{eq:rte:planck}
 \end{equation}
 giving the blackbody radiation for a temperature $T$ and frequency
 $\f$.
  
 If $\sigma$ is constant along the considered part of the LOS, that is, the
 temperature is constant for the case $\sigma=B$, the RTE can be solved
 analytically to give
 \begin{equation}
   I = I_1e^{-\tau} + \sigma\left(1-e^{-\tau}\right)
  \label{eq:rte:step}
 \end{equation}
 where $\tau$ is the optical thickness
 \begin{equation}
   \tau = \int_{l_1}^{l_2}{\kappa(l)dl}
 \end{equation}
 The transmission corresponding to $\tau$ is
 \begin{equation}
   \zeta = e^{-\tau}
 \end{equation}  



\levelb{Practical considerations}
 \label{sec:rte:practical}
 
 The LOS can be divided into parts in several ways. As absorption and
 temperature most likely are avaliable at some vertical grid, the most
 natural choice would be to define the LOS using this vertical grid.
 This solution is problematic for limb sounding as the ratio between
 the distance along LOS and the corresponding vertical distance
 becomes infinite at the tangent point. Another solution would be to
 base the division on $\tau$, but such a division does not guarantee
 that $T$ is close to constant inside the slabs as the vertical
 extension in some cases could be very large, and each combination of
 frequency and viewing angle should require a specific division.
  
 As a practical compromise, it was here decided to divide LOS into
 equal long geometrical steps. With this scheme the division is
 identical for all frequency components, but changes between the
 viewing angles, and should give relatively fast and straightforward
 calculations, maintaining a good accuracy. This approach has
 been applied successfully in the Odin sub-mm forward model 
 \citep{eriksson:97a,eriksson:00a}.
  
 The next question is when and how to calculate LOS and the associated
 variables. As the determination of weighting functions associated
 with the absorption needs basically the same
 quantities as RTE, it is most efficient to do this procedure only
 once and in such way that the values are suitable for both RTE and
 the weighting functions. Hence, the LOS calculations shall be a
 separate part, not included in the RTE functions. The standard use of
 the forward model should then be:
  \begin{enumerate}
    \item Calculation of absorption coefficients.
    \item Determination of LOS.
    \item Calculation of the source function and transmissions along LOS.
    \item Iteration to solve RTE.
    \item Calculation of weighting functions.
    \item Saving etc.
  \end{enumerate}
 The determination of LOS is described separately in Section \ref {sec:los}. 
  

  
\levelb{Practical solution}
 \label{sec:rte:iter}
 
 The LOS is here assumed to be defined with $n$ points where the
 distance between the points is constant (see Fig \ref{fig:rte:los}).
 There are at least two definition points of the LOS ($n\geq1$). The
 absorption and the source function are determined at the points of
 the LOS, and these values are used to calculate the transmission and a
 mean source function value for the distances between the LOS points.
 Only the later two quantities are stored.

  \begin{figure}
    \includegraphics*[width=0.98\hsize]{Figs/los}
    \caption{Schematic description of the LOS and associated variables.
      The absorption and the source function at the LOS points are
      denoted $\kappa_i$ and $\sigma_i$, respectively, while $\zeta_i$
      is the transmission between the points and $\Psi_i$ is the mean
      of neighbouring source function values. Only $\zeta$ and $\Psi$
      are stored for the later calculations. All the points are
      separated by the distance $\Delta l$ (along the LOS). The distance
      between point $i$ and $i+1$ is denoted as step $i$ of the LOS. }
    \label{fig:rte:los}  
  \end{figure}

 \levelc{Absorption and transmission}  
  \label{sec:rte:abs}
  
  The absorption is treated to vary linearly between the LOS points.
  As mentioned above, the transmission values shall be valid between the
  LOS points. With these definitions, the optical thickness associated with 
  step $i$ is
 \begin{equation}
    \tau_i = \frac{\Delta l}{2} \left( \kappa_i+\kappa_{i+1} \right),
                             \quad 1\leq i < n 
  \label{eq:rte:tau}
 \end{equation}
 The relationship between the optical thicknesses and
 the transmission is
 \begin{equation}
   \zeta_i = e^{-\tau_i}
 \end{equation}
 Note that
 \begin{equation}
   e^{-\left(\tau_1+\tau_2\dots\tau_n\right)}=\zeta_1\zeta_2\dots\zeta_n
 \end{equation}
 The absorption at the LOS points is determined from the absorption
 matrix provided by the absorption module by linear interpolation,
 using the logarithm of the pressure as altitude
 coordinate\footnote{The logarithm of the pressure is throughout the
   basic altitude coordinate in ARTS.}.


 \levelc{The source function}  
  \label{sec:rte:source}
  The source function is also basically assumed to vary linearly
  between the LOS points, but for simplicity reasons, a single
  source function value is assigned to the LOS steps:
  \begin{equation}
    \Psi_i = \frac{\sigma_i+\sigma_{i+1}}{2}, \quad 1\leq i < n
   \label{eq:rte:smean}
  \end{equation}   
  The source function at the LOS points ($\sigma$) is simply by
  interpolating linearly the temperature profile, and calculating the
  Planck function (Eq. \ref{eq:rte:planck}) for the obtained
  temperatures.
  
  To fully model that the absorption and the source function have a
  simultanous linear variation between the LOS points would give much
  more complicated analytical expressions than presented here (if even
  possible to derive?). However, the simplified approach used here
  should not influence the accuracy in any important way. This as the
  source function has, compared to the absorption, a relatively low
  variation and it can be treated to be piecewise constant when solving
  the raqdiative transfer.
  
  If long wavelengths are assumed and the source function equals the
  Planck function (Eq.  \ref{eq:rte:planck}), $\sigma$ should
  maximally vary with about a factor of 2 as the minimum and the
  maximum temperature in the atmosphere are about 150 and 300 K,
  respectively, and the relationship between $\sigma$ and temperature
  is close to linear.  This should be compared to the absorption that,
  even for a single frequency, often varies with many orders of
  magnitude.


 \levelc{Solving the radiative transfer equation}  
  \label{sec:rte:solving}
  
  With the definitions given above, the intensity at point $n$ can be
  expressed as
 \begin{equation}
   I = I_1 \prod_{j=1}^{n-1}\zeta_j + 
       \sum_{i=1}^{n-1}\left[\Psi_i(1-\zeta_i)\prod_{j=i+1}^{n-1}\zeta_j\right]
  \label{eq:rte:rteprod}
 \end{equation}
 However, an alternative approach, requiring less computer memory, is
 to follow the radiation from one slab of the atmosphere to next, and
 is the method of choice here. Following Equation \ref{eq:rte:step},
 the following iterative expression can be determined
 \citep{eriksson:97a}
 \begin{equation}
   I_{i+1} = I_i\zeta_i + \Psi_i\left(1-\zeta_i\right)\qquad i=1,2,...,n-1
  \label{eq:rte:iteration}
 \end{equation}
 where $I_i$ is the intensity reaching point $i$.
 The iteration is started by setting $I_1$ to the intensity at the 
 atmospheric limit, that is, cosmic background radiation or correspondingly.


 \levelc{Considering ground reflection}  
  \label{sec:rte:ground}

 The effect of a ground reflection is modeled as
 \begin{equation}
   I^{after} = I^{before}(1-e) + eB(\f,T_{ground})
  \label{eq:rte:ground}
 \end{equation} 
 where $e$ is the ground emission factor and $I^{before}$ and
 $I^{after}$ is the intensity before and after the reflection,
 respectively. See further Section \ref{sec:los:ground}.


\levelb{Optical thicknesses}
 \label{sec:rte:trans}
  
 The atmospheric emission can be neglected if the observation is
 performed towards a sufficiently strong source, such as the Sun, and
 the measurement gives then the total atmospheric transmission,
 $\zeta^{tot}$. When inverting such observations, the standard
 approach is to invert the optical thicknesses $(\tau)$ to obtain a
 more linear inversion problem. For this reason, the output from ARTS
 when neglecting emission was selected to be optical thicknesses
 instead of transmission values. However, as the transmission for each
 step along the LOS is stored for the emission calculations, ARTS
 calculates internally transmission spectra that are converted to
 optical thicknesses.

 This transmission is
 \begin{equation}
   \zeta^{tot} = e^{-\int_{l_1}^{l_2}{\kappa(l)dl}}
  \label{eq:rte:tottrans}
 \end{equation}
 The corresponding iterative formula used in the forward model is
 simply (cf. Eq. \ref{eq:rte:iteration})
 \begin{equation}
   \zeta^{tot} = \prod_{i=1}^{n-1}\zeta_{i}
 \end{equation} 
 It is noteworthy that the multiplication order
 is of no importance, a fact that can be used for 1D limb sounding where
 the conditions are assumed to be symmetrical around the tangent point and
 only one half of the line of sight is stored. 

 If there is a ground reflection, it is considered as
 \begin{equation}
   \zeta^{tot} = (1-e)\prod_{i=1}^{n-1}\zeta_{i}
  \label{eq:rte:tground}
 \end{equation} 
 where $e$ is the ground emission factor. 

 The optical thicknesses are finally calculated as
 \begin{equation}
   \tau^{tot} = -\ln(\zeta^{tot})
  \label{eq:rte:odepth}
 \end{equation} 



\levelb{Cooling rates}
 \label{sec:rte:coolrates}
 
 Cooling rates is an important concept for climate models and studies.
 The cooling rate gives the change in temperature due to exchange of
 radiation, keeping all variables constant. The typical unit is K/day.
 There are two balancing (more or less) effects. Absorption of
 shortwave (UV -- near IR) results in a heating of the air mass. For
 thermal IR, emission and absorption are coupled phenomena but
 exchange of thermal and far IR radiation normally results in a
 cooling effect. These two effects give together the heating rate, but
 are normally calculated seperately and are then denoted as the
 heating and cooling rate. In ARTS only the IR cooling rate is
 handled, and that only for cloud free conditions. For more details on
 heating / cooling rates, see any text book on atmospheric physics.
 
 Cooling rates are calculated by the WSM \verb|CoolingRates|. The
 method returns the spectral cooling rates for the frequencies of
 \verb|f_mono| and the pressure levels of \verb|p_coolrates|. A
 positive value here means a cooling effect (that is, the heating rate
 is not returned). The unit of values in \verb|CoolingRates| is
 K/day/Hz. Directly below follows a derivation of the expression used
 in \verb|CoolingRates|.
 
\newcommand{ \pressure}       {\ensuremath{P}}
\newcommand{ \temperature}    {\ensuremath{T}}
\newcommand{ \altitude}       {\ensuremath{z}}
\newcommand{ \timesymbol}     {\ensuremath{t}}
\newcommand{ \density}        {\ensuremath{\rho}}
\newcommand{ \heatcapacity}   {\ensuremath{c_\pressure}}

\newcommand{ \flux}           {\ensuremath{F}}
\newcommand{ \monoflux}       {\ensuremath{F_\f}}
\newcommand{ \odepth}         {\ensuremath{\tau}}
\newcommand{ \planckfunc}     {\ensuremath{B_\f}}
\newcommand{ \abscoeff}       {\ensuremath{\alpha}}

 The heating rate can be expressed as
 \begin{equation}
  \label{eq:heatrate1}
  \frac{\dd\temperature}{\dd\timesymbol} = 
         \frac{-1}{\density\heatcapacity} \frac{\dd \flux}{\dd\altitude},
 \end{equation}
 where \temperature\ is the temperature, \timesymbol\ is the time,
 \density\ is the air density, \heatcapacity\ is the heat capacitivity
 (for pressure work), \flux\ is radiative flux and \altitude\ is
 vertical altitude. This is the expression normally used to determine
 cooling rates.
 
 The idea here is to find an expression that gives the heating rate if
 the spectral radiance as a function of zenith angle is known.  As a
 first step, Equation~\ref{eq:heatrate1} is rewritten to include the
 spectral flux, \monoflux:
 \begin{equation}
  \label{eq:heatrate2}
  \frac{\dd\temperature}{\dd\timesymbol} = 
       \frac{-1}{\density\heatcapacity} \int_0^\infty \left[
       \lim_{\Delta\altitude\gets 0}\frac{\monoflux(\altitude+\Delta\altitude)-
                \monoflux(\altitude)}{\Delta\altitude} \right] \dd\f.
 \end{equation}
  The relationship between \monoflux\ and spectral radiance,
  \mpbi, is
 \begin{equation}
  \monoflux(\altitude) = 2\pi\int_0^\pi \mpbi(\altitude,\view)
          \cos\view \sin\view \dd\theta.
 \end{equation}
 The assumption below is that the upwelling part of $\mpbi$ is
 known at \altitude\, and the downwelling part $\mpbi$ is known at
 $\altitude+\Delta\altitude$.

 The upwelling spectral radiance at altitude $\altitude+\Delta\altitude$
 can be expressed as
 \begin{equation}
   \mpbi(\altitude+\Delta\altitude,\view) = 
       \mpbi(\altitude,\view) e^{-\odepth(\view)} +
        \planckfunc(1-e^{-\odepth(\view)}),\qquad 0\leq\view\leq\pi/2,
 \end{equation}
 where \planckfunc\ is the Planck function for blackbody radiation.
 The fact that $\Delta\altitude$ will approach zero has been
 used, which mean that \planckfunc\ and \view\ can be assumed to
 be constant between \altitude\ and $\altitude+\Delta\altitude$.
 The downwelling spectral radiance at \altitude\ is in similar way
 \begin{equation}
   \mpbi(\altitude,\view) = 
       \mpbi(\altitude+\Delta\altitude,\view) 
        e^{-\odepth(\view)} +
        \planckfunc(1-e^{-\odepth(\view)}),\qquad \pi/2<\view\leq\pi.
 \end{equation}
 Again using the fact that $\Delta\altitude\approx0$, the optical depth is
 \begin{equation}
   \odepth(\view) = \frac{\Delta\altitude}{|\cos\view|} \abscoeff
 \end{equation}
 and the transmission is
 \begin{equation}
   e^{-\odepth(\view)} = 
      1 - \frac{\Delta\altitude}{|\cos\view|} \abscoeff.
 \end{equation}
 We have then that
 \begin{eqnarray}
   & \frac{\monoflux(\altitude+\Delta\altitude)- \monoflux(\altitude)}
       {\Delta\altitude} = \frac{2\pi}{\Delta\altitude}\Big\{
     \int_0^{\pi/2} \left[ \mpbi(\altitude,\view)
          (1 - \odepth(\view)) + \planckfunc\odepth(\view)\right]
          \cos\view \sin\view \dd\view + & \nonumber \\
   & + \int_{\pi/2}^\pi \mpbi(\altitude+\Delta\altitude,\view)
          \cos\view \sin\view \dd\view - 
       \int_0^{\pi/2} \mpbi(\altitude,\view)
          \cos\view \sin\view \dd\view - & \nonumber\\   
   & -\int_{\pi/2}^\pi \left[ \mpbi(\altitude+\Delta\altitude,\view)
          (1 - \odepth(\view)) + \planckfunc\odepth(\view)\right]
          \cos\view \sin\view \dd\view  \Big\}. &
 \end{eqnarray}
 Many of the terms above cancel out and the expression above can be
 shortened to
 \begin{eqnarray}
   \label{eq:diffF}
   & \frac{\monoflux(\altitude+\Delta\altitude)- \monoflux(\altitude)}
       {\Delta\altitude} = 2\pi\Big\{
     \int_0^{\pi/2} \left[ \planckfunc -\mpbi(\altitude,\view)
           \right] \abscoeff \sin\view \dd\view + & \nonumber \\
   & +\int_{\pi/2}^\pi \left[ \planckfunc -
                \mpbi(\altitude+\Delta\altitude,\view)
          \right]\abscoeff \sin\view \dd\view  \Big\}. &
 \end{eqnarray}
 Putting Equation~\ref{eq:diffF} into Equation~\ref{eq:heatrate2}, and
 noting that
 $\mpbi(\altitude,\view)=\mpbi(\altitude+\Delta\altitude,\view)$
 for $\Delta\altitude=0$,  gives finally
 \begin{equation}
   \label{eq:heatratefinal}
    \frac{\dd\temperature}{\dd\timesymbol} = 
        \frac{-2\pi}{\density\heatcapacity} \int_0^\infty 
     \int_0^\pi \left[ \planckfunc -\mpbi(\altitude,\view)
           \right] \abscoeff \sin\view \dd\view   \dd\f.
 \end{equation}
 We can already here note that the contribution to the heating rate
 will be zero for frequencies where $\abscoeff=0$.
 
 To obtain correct results it is crucial that \mpbi\ converges to
 \planckfunc\ when it is expected that $\mpbi=\planckfunc$, which is the
 case when the absorption is very high. Considering that radiative
 transfer applies a mean value of the Planck function at the end
 points of the integration step (Eq.~\ref{eq:rte:smean}), it is not a
 good idea to compare \mpbi\ with the \planckfunc\ for the position of
 interest. This would require an extremly short radiative transfer
 step length (\verb|l_step|), a statement verified by practical
 calculations. A better solution is to replace $(\planckfunc
 -\mpbi(\altitude,\view))$ in Equation~\ref{eq:heatratefinal} by
 $(\Psi-\mpbi(\altitude,\view))$, where $\Psi$ (defined in
 Eq.~\ref{eq:rte:smean}) is the effective source function for
 radiative transfer step closest to the point of interest. This
 modification will balance the radiation budget perfectly for high
 values of $\abscoeff$ and has a small impact on the accuracy.
 However, as for all radiative transfer calculations in ARTS, reducing
 the step length will improve the calculation accuracy.
 
 

 \levelb{Control file examples}
  \label{sec:rte:cfe}
 
  See Section \ref{sec:los:cfe}.



%%% Local Variables: 
%%% mode: latex
%%% TeX-master: "main"
%%% End: 
