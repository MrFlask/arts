%
% To start the document, use
%  \levela{...}
% For lover level, sections use
%  \levelb{...}
%  \levelc{...}
%
\levela{Data reduction}
 \label{sec:red}


%
% Document history, format:
%  \starthistory
%    date1 & text .... \\
%    date2 & text .... \\
%    ....
%  \stophistory
%
\starthistory
  000321 & Created and written by Patrick Eriksson.\\
\stophistory


%
% Symbol table, format:
%  \startsymbols
%    ... & \verb|...| & text ... \\
%    ... & \verb|...| & text ... \\
%    ....
%  \stopsymbols
%
%
%\startsymbols
%  -- & -- & -- \\
% \label{symtable:red}     
%\stopsymbols



%
% Introduction
%
Many observation scenarios give rise to very large measurement
vectors, larger than can be handled practically during the inversions,
and some kind of reduction of the data size is needed. This data
reduction can be made part of the sensor transfer matrix. In fact, the
data reduction can be viewed upon as an imaginary second spectrometer.
The transfer matrix to use is then (Eq. \ref{eq:formalism:Hs})
\begin{eqnarray}
  \Hm = \Hd \Hs  \nonumber
\end{eqnarray}
where \Hd\ is the data reduction matrix and \Hs\ the sensor matrix.
{\it Data reduction can so far only be performed in Qpack.}


\levelb{Averaging of viewing angles}
 \label{sec:red:view}
 
 In some cases the spectra from different viewing angles are combined,
 either as a pure data reduction or internally in the spectrometer.
 The rows of \Hd\ for this case have the structure
 \begin{equation}
   \mat{h} = \big[ 0,\dots,0,\frac{1}{n_v},0,\dots,0,\frac{1}{n_v},0,\dots,0,\frac{1}{n_v},0,\dots,0\big]
 \end{equation}
 where $n_v$ is the number of viewing angles to combine.


\levelb{Data binning}
 \label{sec:red:binning}
 
 Data binning means that neighboring channels are combined by
 weighted averaging. If channels $i_1$ to $i_2$ of $\y'$ are combined to
 give element $j$ of $\y$, the binning can be expressed as
 \begin{equation}
   \y^j = \frac{1}{\sum_{i=i_1}^{i_2}{\Delta \f^i}} \sum_{i=i_1}^{i_2}{\Delta \f^i (\y')^i}
 \end{equation}
 Row $j$ of \Hd\ is accordingly
 \begin{equation}
   \mat{h}^i = \frac{\Delta \f^i}{\sum_{i=i_1}^{i_2}{\Delta \f^i}}, \qquad
    i_1\leq i \leq i_2
 \end{equation}
 Other values of $\mat{h}$ are zeros. The matrix \Hd\ is for data
 binning highly sparse.



\levelb{Reduction by eigenvectors}
 \label{sec:red:eig}
 
 A commonly used approach for reducing data sizes is to base the
 reduction of the eigenvectors of the covariance matrix expressing the
 variability of the measurements. These empirical eigenvectors
 fulfills the relationships
 \begin{equation}
   \mat{S}_\y = \mat{E}\Lambda\mat{E}^T
 \end{equation}
 where $\Lambda$ is a diagonal matrix holding the eigenvalues
 corresponding to the eigenvectors, the columns of $\mat{E}$. The
 eigenvectors form an orthogonal basis:
 \begin{equation}
   \Id = \mat{E}^T_j\mat{E}_j
 \end{equation}
 where $\mat{E}_j$ signifies the $j$ first columns of the matrix.

 The data reduction for this case is performed as
 \begin{equation}
   \y = \mat{E}^T_j \y'
 \end{equation}
 that is
 \begin{equation}
   \Hd = \mat{E}^T_j 
 \end{equation}
 By basing the data reduction on the covariance matrix eigenvectors,
 the reduction maintaining the maximum possible fraction of the
 variability of the spectra, for a given $j$, is achieved.

 Different versions of this scheme are described in \citet{eriksson:01c}. 
 The existing options in Qpack are described in the file \verb|README|.





%%% Local Variables: 
%%% mode: latex
%%% TeX-master: "uguide"
%%% End: 
